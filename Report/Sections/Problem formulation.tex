\begin{figure}
    \centering
    \includegraphics[width=0.47\textwidth]{images/Scheme_drone_tag.png}
    \caption{Application model scheme, in which the actual planar range $\rho^*$, the angles and both the drone local and the global reference frames, are highlighted. In blue is represented the double UWB antenna, deployed perpendicular to the drone's x direction.}
    \label{PRFOR:fig:dronetag_scheme}
\end{figure}


As previously introduced, the problem addressed in this work, is the classical \textit{leader-follower} application using relative range and angle. The policy to be maintained by the follower during the leader motion, is to stay at a constant distance (in the x-y plane), with an AoA of zero. The target (leader) is tracked using the single UWB (i.e. the tag), whereas the quadrotor is equipped with the double UWB (i.e. the follower). Let us define the coordinates of both tag \textbf{p} and follower \textbf{d}, in the global reference frame
\begin{equation}
\textbf{d} = \begin{bmatrix} x_d,y_d,z_d\end{bmatrix}^T \quad , \quad \textbf{p} = \begin{bmatrix} x_p,y_p,z_p\end{bmatrix}^T
\end{equation}
As said, the quantity delivered by the sensor are the range between the two antennas \textbf{$\rho$}, and the angle of arrival of the tag w.r.t. the follower \textbf{$\alpha$}. The range can be expressed as
\begin{equation}
    \rho = ||\mathbf{d} - \mathbf{p}||=\sqrt{(x_d - x_p)^2 + (y_d - y_p)^2 + (z_d - z_p)^2} + \eta = \Bar{\rho} + \eta
\end{equation}
Where $\eta$ is the measure uncertainty, which is supposed to be normally distributed, zero-mean and white, i.e $\eta \sim \mathcal{N}(0,\sigma_{\rho}^2)$. The variance of this quantity, is related to the UWB signal bandwidth and of course with the actual distance $\bar{\rho}$\cite{uwb_variance}. The range measurement is also affected by a bias introduced by the timestamp delay, which is different for each type of radio and is function of the actual distance $\bar{\rho}$\cite{UWBEvaluationOP}. Since in the proposed application the measured distance (without considering the initial conditions), should stay in a relatively strict band, a calibration in that band of interest, should be enough to mitigate the bias. Given that, the bias is supposed to be negligible. In the end, Line-of-sight (LOS) and negligible shadowing conditions are assumed, and so also Non-line-of-sight and shadowing sources of error\cite{UWBHumshadowing}, are neglected.\\
Clearly the measure obtained is the pure 3D range, which is not the wanted quantity, since the described policy is to maintain a certain distance in x-y plane. With the described configuration, it is not possible to directly obtain this value. But if we suppose that the quadrotor can collects measurements about its height $h$ w.r.t. to the ground (e.g. with a barometer, or any type of global positioning system) and assuming a fixed known value for the target height, is it possible to calculate the projection of the range in the x-y plane $\rho^*$. The assumption about the target height, can be simply made by knowing at which height the tag will be kept. Since in our test the tag is placed on a ground robotic agent, we have simply measured the height and saved as the constant $h_t$. In the end the projection $\rho^*$ can be calculated as follow:
\begin{equation}
    \rho^* = \sqrt{\rho^2 - \big( z_d - h_t\big)^2}
\end{equation}
where the double UWB antenna height $z_d = h + h_d$, i.e. the sum of the quadrotor height estimates $h$ and the relative height of the double antenna w.r.t. the flight controller $h_d$ (which is a construction constant).\\

For what concerns the angle of arrival, if we consider the double UWB face placed perpendicular to the x drone axis (as depicted in \autoref{PRFOR:fig:dronetag_scheme}) , it is possible to define the angle of arrival \textbf{$\alpha$} measured by the UWB, as follow:
\begin{equation}
    \alpha = \theta -\psi+ \xi
\end{equation}
where $\psi$ is the yaw orientation of the quadrotor, $\theta$ is the the geometric angle between the follower and the tag in the x-y plane and $\xi$ is the measure uncertainty, which, as for the range, is supposed to be normally distributed, zero-mean and white, that is $\xi \sim \mathcal{N}(0,\sigma_a^2)$. The same consideration and assumption done for the range, are also valid for the angle measure. This angle is obtained from the PDoA, that is the difference in phase of the signal received by the two antennas of the follower, by means of the following relation:
\begin{equation}\label{PRFOR:eq:aoa-pdod}
    \alpha = \arcsin \Bigg( \frac{\Delta\phi \cdot \lambda}{2\pi\cdot d} \Bigg)
\end{equation}
where $\Delta\phi$ is the PDoA, $\lambda$ is the signal wavelength and $d$ is the wheelbase between the two, in the double UWB, antennas. This conversion from PDoA to AoA is performed directly inside the Nordic companion board of the antenna, given that all the constant like $\lambda$ and $d$ are given in construction phase. Moreover, as will be later explained, this conversion is performed by knowing the calibration PDoA offset and the number of element over which averaging it. Both those quantities can be modified from serial port, to obtain the better possible value in the calibration phase.

\subsection{Control law}\label{control_law}
Let us define the wanted policy, as a vector with the wanted range, angle and height $r_w = \begin{bmatrix} \rho_w, \alpha_w, z_w\end{bmatrix}^T$ and the actual range, angle and height at time t $r_a(t) = \begin{bmatrix} \rho^* (t) , \alpha(t), z(t) \end{bmatrix}^T$. The control law has the scope of minimize and maintain low the difference $r_a - r_W$ in time. To do so three different control strategies are implied:
\begin{itemize}
    \item a PID controller to generate the x-y velocities setpoints
    \item a setpoint control for the yaw
    \item a setpoint control for the height
\end{itemize}
The PID controller is based on the difference between the actual measured range $\rho^*$ and the wanted value $\rho_w$, and by defining $d = \rho^* - \rho_w$, a scalar control value $V_s$ can be calculated:
\begin{equation}
    V_{s_k} = K_p\cdot d_k + K_d\cdot\Big(\frac{d_k - d_{k-1}}{\Delta t}\Big) + K_i\cdot I_{d_k}\Delta t
\end{equation}
Where $K_p$, $K_d$, $K_i$ are the proportional, derivative and integral gain respectively, $\Delta t$ is the time-step and $I_d$ is the cumulative integral value which can be defined as follow:
\[
    I_{d_k} = I_{d_{k-1}} + d_k
\]
The subscript $k$ defines the discrete $k$ instant at which the control is calculated. In order to obtain the actual x-y pair velocities, a last projection step has to be performed. Since the quantity $V_{s_k}$ represent the velocity to be applied in the direction of the target, the $V_{x_k}$ and $V_{y_k}$ control velocity, expressed in the global frame, can be obtained as follow:
\begin{equation}\label{PF:VELxy}
    \begin{bmatrix} V_{x_k} \\ V_{y_k} \end{bmatrix} = V_{s_k} \cdot \begin{bmatrix} \cos (\theta_k) & \sin (\theta_k) \end{bmatrix}^T
\end{equation}
where $\theta_k$, as already mentioned, is the geometric planar angle between the target and the drone and can be derived as:
\begin{equation}\label{PF:yawsp}
    \theta_k = \alpha_k +\psi_{k_{estim}}
\end{equation}
where $\psi_{k_{estim}}$ is the quadrotor yaw angle, estimated by the flight controller. Obviously this quantity is affected by both the uncertainty of the AoA measure $\alpha_k$ and of $\psi_{k_{estim}}$.\\
The two obtained control velocity, are fed to the flight controller as a velocity setpoint $V_{sp} = \begin{bmatrix} V_{x_k}, V_{y_k}, *\end{bmatrix}$, whereas $\theta$ is fed as the yaw setpoint, $\psi_{sp}$, to orient the drone toward the target. Lastly the height is controlled by feeding the flight controller with a position setpoint, $X_{sp} = \begin{bmatrix} *, *, z_w\end{bmatrix}$\footnote{The * simply means that the quantity is not given so not controlled}.
 




% In this leader-follower target problem, the leader can be anything movable entity while the follower is a quadcopter. The follower has the task of following the movements done by the leader, keeping a constant distance $r_{foll}$ in the plane x-y and a null angle $\theta_{foll}$, always in the same plane, w.r.t. the target. In order to do this the quadcopter needs to localize the target. We assume that the drone is equipped with all the sensors required to fly: GPS, IMU, compass, barometer as well with a flight controller that can properly fuse and filter all the sensor informations to localize itself globally and that can take as input setpoints (in position or velocity) and control the propellers to reach them. To localize the target globally is therefore sufficient a relative positioning system w.r.t. the follower, if its global pose it is known. For the relative positioning, we use a pair of UWB modules, and the one mounted on the drone has a pair of antennas while the one on the target is single. With this setup the planar azimuth angle can be estimated in addition to the ranging distance. The performance and the characterization of the UWB system will be discussed successively. \\
% With these informations, the PID control law that we adopted to compute the proper velocity setpoints expressed in the global frame to be sent periodically to the flight controller are calculated in this way:
% \begin{equation}\label{MOD:velocities}
% \begin{bmatrix}
%     v_{x} \\
%     v_{y}
% \end{bmatrix}
% = (k_{P} \cdot r_{P} + k_{D} \cdot r_{D} + k_{I} \cdot r_{I}) \cdot 
% \begin{bmatrix}
%     cos(\psi_{set})\\
%     sin(\psi_{set})
% \end{bmatrix}
% \end{equation}

% $k_{P}$, $k_{D}$ and $k_{I}$ are the chosen PID gains, while $\psi_{set}$ is the yaw commanded in order to orientate the front of the quad toward the target: 

% \begin{equation}\label{MOD:yaw}
%     \psi_{set} = \psi_{glob} + \theta_{UWB}
% \end{equation}


% \begin{equation}\label{MOD:controlPID}
% \begin{bmatrix}
%     r_{P}\\
%     r_{D}\\
%     r_{I} 
% \end{bmatrix} =
% \begin{bmatrix}
%     r_{xy} - r_{foll}\\
%      \frac{r_{P} - r_{P prev}}{dt}\\
%     r_{I} + r_{P} dt
% \end{bmatrix}
% \end{equation}
% The subscript UWB indicates informations that comes from the sensors, $dt$ indicates the period with which setpoints are sent and $\psi_{glob}$ is the estimated yaw in the global frame. The velocity in the z-axis is not reported since we assume that the the flight takes place at a constant height, $z_{fly}$ reached after a takeoff and the flight controller is able to maintain it. It is even assumed that the height of the tag $z_{tag}$ on the target does not change, so $r_{xy} = \sqrt{r_{UWB}^2 - (z_{fly}-z_{tag})^2}$. This control law will be tested in simulation and with real hardware to understand if it can be suitable and to evaluate its performance.


% Let us define the wanted policy, as a vector with the wanted range, angle and height $r_w = \begin{bmatrix} \rho_w, \alpha_w, z_w\end{bmatrix}^T$ and the actual range, angle and height at time t $r_a(t) = \begin{bmatrix} \rho^* (t) , \alpha(t), z(t) \end{bmatrix}^T$. The control law has the scope of minimize and maintain the difference $r_a - r_W$ in time as wanted, flying at constant height and keeping the quadcopter orientation pointed toward the target.\\ To do so three different control strategies are implied: