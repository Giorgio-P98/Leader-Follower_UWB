Our work, through simulations and real tests, has demonstrated and validated the functioning of the developed solution, and the potential of this new type of UWB in the field. Nevertheless, it is clear that there is still room for improvements. In particular, other refinements on the UWB tuning and on its firmware, could provide better target pose estimation and consequently better performances. Indeed, as suggested in \autoref{UWB_CHAR}, modelling the non-linearity of the PDoA w.r.t. the distance, for example developing a proper look up table, could improve the performances and the robustness. From the manufacturing point of view, the development of a pure isotropic target antenna, can improve the measurement performance to, solving the antennas mutual orientation related problems pointed in \autoref{UWB_CHAR} conclusions.\\

For what concerns our strategy, we directly use the measures without filtering and fusion. Investigation in filtering methods that also consider prior information about the motion of the target, are suggested to improve the strategy robustness. Also our PID controller could be better tuned, performing some optimization on the gains.