\begin{figure}
    \centering
    \includegraphics[width=0.2\textwidth]{images/characterization/QM33120WDK1_PDP_v2.png}
    \caption{QM33120WDK1 Ultra-Wideband (UWB) Transceiver Development Kit. The Nordic nRF52840DKs (the blue boards behind), are the actual companion boards on which the firmware runs.}
    \label{SEN:fig:eval_kit}
\end{figure}

As already underlined, for the target position estimate, we have used the Ultra-Wideband (UWB) Transceiver Development Kit produced by Qorvo \autoref{SEN:fig:eval_kit}, for which we refer to the official website page \cite{UWBQorvo} and datasheet \cite{UWBDatasheet} for details. For this kit, is also available a firmware that can be requested from Qorvo. Reading the code, we have deduced the following main information:
\begin{itemize}
    \item AoA and distance, are fed directly to the serial port in an encoded string with other quantity like PDoA, status and block number;
    \item the firmware can run in different configurations, depending on the antenna type and the selected communication channel\footnote{The channel type, allows to change between the two supported IEEE standards: IEEE Std. 802.15.4™‐2020\cite{IEEEstd_4} and IEEE Std. 802.15.4z™‐2020\cite{IEEEstd_4z}}: \textit{Jolie} and \textit{Monalisa} are the two antenna types. The possible channels are instead $9$ and $5$. It follows that the antenna can be configured in $4$ different ways, using the parameter \texttt{ANTENNA}: \texttt{JOLIE9}, \texttt{JOLIE5}, \texttt{MONALISA9}, \texttt{MONALISA5}. Our antenna is a \textit{Jolie} and the default channel is $9$, meaning that the \texttt{ANTENNA} parameter is equal to \texttt{JOLIE9} by default;
    \item PDoA raw measures, are adjusted by means of a stepped linear regression. The steps of this linear regression are defined by a pair of look up table (different for each channel and antenna type): one for the PDoA intervals, and one for the slopes that the liner regression model has to have inside these intervals. All the values were clearly calculated after laboratory test, aiming to reduce the non linear effects present in the raw PDoA measure; 
    \item the PDoA value truly used to calculate AoA by means of the relation \eqref{PRFOR:eq:aoa-pdod}, is the average of the last \texttt{PAVRG} values, corrected with \texttt{PDOAOFF}. \texttt{PAVRG} and \texttt{PDOAOFF} are user definable constants by default equal to $10$ and $0$ respectively. The first defines the values over which perform the moving average: an high value can mitigate the high frequency error, but can worst the sensor performance in detecting PDoA changes; on the other side a lower value can detect even rapid changes in the actual PDoA, but the high frequency error contribution is not lowered. \texttt{PDOAOFF} defines instead the offset to apply at the average result as a user calibration.
\end{itemize}

\begin{figure}
    \centering
    \includegraphics[width=0.45\textwidth]{images/characterization/PDOAOFF_trend.png}
    \caption{\texttt{PDOAOFF} at various distances.}
    \label{UWB:fig:PDOAFF}
\end{figure}

\subsection{Characterization}
In order to understand the behavior of the kit, we have performed a sensor characterization to asses what follows:
\begin{itemize}
    \item Performance difference using the two supported IEEE standards (i.e. channel $9$ or $5$), comparing the results at different angles and distances;
    \item Performance at different tag heights and angles;
    \item Performance putting the tag behind the double-UWB anchor;
\end{itemize}
Since the first tested channel configuration was the default one (i.e. $9$), the last two results were extracted only for this configuration. By the way, channel type should not change noticeably the validity of the extracted information (e.g. if the kit performance are worsen by a tag height change, the channel should not be a discriminant).\\
All the listed assessment, where performed using MoCap as ground truth and even for precise positioning (i.e. to place the tag at precise distances, angles and eventually heights w.r.t the double UWB).\\

\begin{figure*}
    \centering
    \includegraphics[width=1.0\textwidth]{images/characterization/ch9_characterization.png}
    \caption{Location test conducted with channel 9 standards, expressed in centimeters.}
    \label{UWB:fig:ch9test}
\end{figure*}

\begin{figure}
    \centering
    \includegraphics[width=0.45\textwidth]{images/characterization/ch9_aoa_hist.png}
    \caption{Channel 9 AoA measurements distribution at $-15$ degrees and $200$ centimeters.}
    \label{UWB:fig:AoA_ch9}
\end{figure}

Since the UWB kit declared working region is the half-plane in front of the double UWB, to compare the behaviour under the two different IEEE standards, we have perform tests in this region, at different discrete values of range and angles. Given that this half-plane can be defined with a polar coordinate pair, w.r.t. the double antenna, with undefined radius and angles from $-90$ to $+90$ degrees (being $0$ degrees when the the tag is in front of the double, with angle positive in clockwise sense), we have performed the tests as follow:
\begin{itemize}
    \item For channel $9$, at three different range $\rho_9=\begin{bmatrix} 100,200,300 \end{bmatrix}$ cm and angles from $-90$ to $+90$ degree with a step of $15$;
    \item For channel $5$, at two different range $\rho_5=\begin{bmatrix} 150, 250 \end{bmatrix}$ cm and angles from $-90$ to $+90$ degree for the first range and from $-60$ to $+60$ degree for the second, again with a step of $15$.
\end{itemize}

As suggested by the manufacturer, before the data collection we have calibrated the PDoA offset placing the modules at a distance of at least 1.5 meters at 0 degrees, collecting 1000 measures of the PDoA, averaging them to estimate the \texttt{PDOAOFF} value to set. For the channel 9 configuration, we determined it at a distance of $2.5$ meters, thinking that the range did not have much influence. However, before the channel 5 data collection, we realized that PDoA offset, changes with distance, as can be seen in \autoref{UWB:fig:PDOAFF}. Hence, for channel 5, the PDoA calibration was performed at a distance of $1.5$ meters, which is congruent with the following policy for the indoor test.\\

\begin{figure}
    \centering
    \includegraphics[width=0.45\textwidth]{images/characterization/ch9_range_hist.png}
    \caption{Channel 9 range measurements distribution at $-15$ degrees and $200$ centimeters}
    \label{UWB:fig:Range_ch9}
\end{figure}

\begin{figure*}
    \centering
    \includegraphics[width=1.0\textwidth]{images/characterization/ch5_characterization.png}
    \caption{Location test conducted with channel 5 standards, expressed in centimeters.}
    \label{UWB:fig:ch5test}
\end{figure*}

\begin{figure}
    \centering
    \includegraphics[width=0.45\textwidth]{images/characterization/ch5_aoa_hist.png}
    \caption{Channel 5 AoA measurements distribution at $+15$ degrees and $150$ centimeters.}
    \label{UWB:fig:AoA_ch5}
\end{figure}

\paragraph{Channel Comparison}
The results for channel 9, are depicted in \autoref{UWB:fig:ch9test}. The graph shows different behaviours at different distances. For example at a distance of $3$ meters the estimated angle is quite good, whereas at $2$ meters it shows different behaviours for the left and the right part of the half-plane. Indeed the left part shows better angle measures then the right one. Despite this, the measured angles are generally bad. This behaviour could be caused by the great variation of \texttt{PDOAFF} at $2$ meters w.r.t. $2.5$, as shown in \autoref{UWB:fig:PDOAFF}. At $1$ meter the results are slightly better, but still there is asymmetry between positive and negative part. Another characteristic  to denote is that, despite the distance, at angles greater then $75$ degree (and $-75$), the measurements are not reliable and tend to be much more scattered. For what concerns the range, a not constant negative bias is always present. The data clouds, shows instead an almost constant spread, indicating that the standard deviation of both range and angles, are constants. A pair of distribution examples, are presented in \autoref{UWB:fig:AoA_ch9}, \ref{UWB:fig:Range_ch9}. Both shows a bias w.r.t. the actual angle and range, but their distributions are gaussian-like, with $\sigma_{\alpha} = 1.79$ ° and $\sigma_{\rho} = 3.05$ cm and mean error of $3.30$ ° for the angle and of $15.47$ cm for the range. Comparing these values with the manufacturer's specifications, reported in \autoref{UWB:tab:sensorspec}, is possible to see that the standard deviation is in accordance with the declared one both for the angle and the range. For what concerns the accuracy, the angle is inside the interval declared while it is not the case for the range. However, as can be seen in \autoref{UWB:fig:ch9test}, if instead of the measurement at $2$ meters and $-15$ degrees one would consider the data at the same distance and angle mirrored, the angle would not be anymore in the accuracy interval.\\

\begin{table}
    \centering
    \caption{Manufacturer declared Location Accuracy Characteristics. See Datasheet p.17 for reference\cite{UWBDatasheet}}
    \begin{tabular}{ | m{1.5cm} | m{1.5cm}| m{2.1cm} | m{0.7cm} |} 
        \hline
        &  accuracy & standard deviation & Units\\ 
        \hline
        Range & $\pm $ 6 & 3 & cm\\ 
        \hline
        AoA CH9 &$\pm$ 6.25 & 2.5 & deg \\
        \hline
        AoA CH5 &$\pm$ 6.25 & 2 & deg \\
        \hline
    \end{tabular}
    \label{UWB:tab:sensorspec}
\end{table}

The test for the channel 5, is depicted in \autoref{UWB:fig:ch5test}. As previously specified, this test was conducted by tuning the PDoA offset at a distance congruent with the one of the following task, that is $1.5$ m, in order to reduce the error in the working region. To this end, even the tested ranges were changed to a pair $1.5$ and $2.5$ m.\\
Also for this test, the asymmetry between negative and positive region is clear. Despite the slightly lower accuracy in range, the left region, shows better performance in angle measurements. Overall the general accuracy is visibly better for this configuration, and more precisely in the cone from $-15$ to $+15$ degrees, the localization is very good. This is an important results since, as will be later shown, during the simulation the angle measured by the simulated UWB remains almost always inside this region. Two representative distribution for angle \autoref{UWB:fig:AoA_ch5} and range \autoref{UWB:fig:Range_ch5} measurements are presented. Their distributions are gaussian-like, with $\sigma_{\alpha} = 1.34$ ° and $\sigma_{\rho} = 1.66$ cm and mean error of $0.4$ ° for the angle and of $0.68$ cm for the range. Comparing this results with the declared specifications for channel 5 \autoref{UWB:tab:sensorspec}, both accuracies and standard deviations are congruent. In particular the accuracy error, both for range and Aoa, is very low.\\
As an overall conclusion, channel 5 outperformed channel 9, making it our choice for the presented application.\\

\begin{figure}
    \centering
    \includegraphics[width=0.45\textwidth]{images/characterization/ch5_range_hist.png}
    \caption{Channel 5 range measurements distribution at $+15$ degrees and $150$ centimeters}
    \label{UWB:fig:Range_ch5}
\end{figure}

\paragraph{Performance for height change}
in our Leader-follower application, the angle estimation robustness against the height difference between leader and follower, is an important requirement. Indeed, the target is supposed to be a ground robot, so the tag is placed at a low height. To asses the kit performance at different heights, we have placed the target antenna at 3 different heights $h=[46,60,115]$ centimeters and at different angles $\alpha = [0,-30,-60,-75]$ degrees. The results are exposed in \autoref{UWB:fig:ch9_heights}. It can be seen that the height difference does not remarkably change the measured angle. Only for the measurements at $-75$ degree the measures are bad but, as previously stated, this is an angle related behaviour, so it does not invalidate the test's outcomes.\\

\begin{figure}
    \centering
    \includegraphics[width=0.48\textwidth]{images/characterization/ch9_heights.png}
    \caption{Angle measurements at different heights and at a constant x-y distance of $200$ cm.}
    \label{UWB:fig:ch9_heights}
\end{figure}

\paragraph{Behind the double UWB results}
in order to fully understand the sensor behaviour in different situations, a test with the target antenna behind the double at a distance of $1$ meter was conducted. The results, together with the equivalent measures at one meter taken in front, are presented in \autoref{UWB:fig:ch9_behind}. As expected, given the manufacturer indication, the measures behind are quite chaotic, and both angles and ranges are mostly wrong. Even the data clouds are spread differently, not resembling the constant behaviour seen for the front taken measures.\\

\begin{figure}
    \centering
    \includegraphics[width=0.48\textwidth]{images/characterization/ch9_behind_measure.png}
    \caption{Localization comparison between front and behind taken measures, expressed in centimeters.}
    \label{UWB:fig:ch9_behind}
\end{figure}

\paragraph{Conclusions}\label{UWB:conclus}
This evaluation kit has the potential to became a valid tool in Leader-follower applications, but some extra work on the sensor is needed. The main drawback is that the angle offset is a non linear function of the distance, and for application in which a dynamic range change is involved, this UWB pair is not suitable. To compensate for this behaviour, some work in modeling this offset's non-linearity is suggested. For what concern our purposes, this kit is precise enough, since in the region of interest (i.e. at $1.5$ meters $\pm 15$ degrees), it works pretty well. Even in the case of having wider angle and distances, since localization maintain his sign (e.g. if the true angle is $60$ degree, it measure $45$ not $-45$), our policy tends to converges towards the $0$ degree angle and $1.5$ meters distance. Therefore, even in the presence of large accuracy error, the quadcopter is still somehow able to perform decently. Another modification effect on the measures noticed, that is not presented in this characterization, is the change in localization given by the relative orientation between the antennas. Indeed the best measures are obtained if the tag is placed radially w.r.t. the double, whereas important angle errors can be introduced if this condition is not respected (especially at short distances). Probably this error is due to the non isotropic tag antenna. This can overall truly impact the localization performances. In the real test, whose results will be later exposed, this last behaviour is surely responsible for a great amount of the localization error.